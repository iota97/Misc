\documentclass[12pt, a4paper]{article}
\usepackage[utf8]{inputenc}
\usepackage[italian]{babel}
\usepackage{fancyhdr}
\usepackage{datetime}

\newdateformat{monthyeardate}{\monthname[\THEMONTH] \THEYEAR}

\def\labelitemi{--}
\setcounter{tocdepth}{2}
\pagestyle{fancy}
\fancyhf{}
\renewcommand{\headrulewidth}{1pt}
\renewcommand{\footrulewidth}{1pt}
\lhead{RICETTE}
\rhead{\leftmark}
\rfoot{Pagina \thepage}

\title{Ricette di cucina}
\author{Giovanni Cocco \thanks{ricette da varie fonti}}
\date{\monthyeardate\today}

\begin{document}

\begin{titlepage}
\maketitle
\thispagestyle{empty}
\end{titlepage}

\begin{abstract}
In questo ricettario verranno collezionate ricette di cucina imparate negli anni.
\end{abstract}
\clearpage

\tableofcontents{}
\clearpage

\section{Primi piatti}

\subsection{Risotto}

Il classico risotto

\subsubsection{Ingredienti}
\begin{itemize}
\item 1 cipolla o scalogno
\item 80 gr di riso vialone nano a testa
\item 10 gr di parmigiano a testa
\item 5 gr di burro a testa
\item 1 dado o un cucchiaino di brodo granulato
\item 10 gr di olio di oliva
\end{itemize}

\subsubsection{Preparazione}
	Sciogliere il dado o il brodo granulato in un pentolino d'acqua e
	portarlo a bollore.\\\\
	Tritare la cipolla o scalogno e buttarla assieme all'olio in una 
	pentola, soffriggere la cipolla (lasciare cuocere nell'olio finchè 
	non cambia colore mescolando di tanto in tanto).\\\\
	Buttare il riso nel soffritto, mescolare e aggiungere il brodo caldo.
	Continuare a mescolare aggiungendo brodo di tanto in tanto per 
	tenere il livello per 10 minuti.\\\\
	Aggiungere il burro e il armiggiano e cuocere per altri 3-7 minuti,
	aggiungendo brodo o aumentando il fuoco per ottenere la giusta 
	consistenza al termine della cottura.

\clearpage
\subsubsection{Varianti}
\begin{itemize}
\item \emph{Piselli} (70 gr a testa sgocciolati) buttarli assieme al riso,
	meglio di dimensioni medie o grandi.
\item \emph{Gorgonzola} (30 gr a testa) sostituisce il burro, ma è molto
	più liquido e tende ad allungare il risotto.
\item \emph{Zafferano} aggiungere a 2 minuti dalla fine.
\item \emph{Salsa di pomodoro} (100 gr a testa) buttare assieme al riso.
\item \emph{Salsiccia} (1/2 a testa) togliere la pelle e tritarla, buttarla 
	assieme	alla cipolla.
\item \emph{Speck} (50 gr a testa) scaldare in un pentolino a parte per
	abbrustorilo un po' e poi aggiungere assieme al burro.
\item \emph{Zucchini} (1 a testa) cuocere in una pentola con un po, d'acqua
	per 30 minuti per sfaldarli per bene, poi aggiungere 
	assieme	al riso.
\end{itemize}
\clearpage

\section{Secondi piatti}

\subsection{Spezzatino}

Spezzatino di vitello con polenta

\subsubsection{Ingredienti}
\begin{itemize}
\item  \.2 kg di muscoletti di vitello
\item 	1 kg di patate
\item 	4 cipolle
\item 	2 carote
\item 	1 gambo di sedano
\item 	salvia e rosmarino
\item 	500 gr di salsa di pomodoro
\item 	sale e pepe
\item 	1 dado
\item 	100 gr di farina 00
\item 	15 gr di olio d'oliva
\item 	500 gr di farina gialla
\end{itemize}

\clearpage
\subsubsection{Preparazione}
	Tritare cipolle, carote e sedano, metterli in una teglia grande
	e mettere a cuocere sul fornello.\\
	Dopo 30 minuti aggiungere l'olio, il dado, la salvia e rosmarino
	tritati e mescolare fino a scioglimento del dado.\\\\
	Infarinare la carne col la farina 00 e metterla nella teglia.
	Aggiungere il pomodoro, il sale e il pepe.\\
	Infornare a forno statico 180 gradi.\\\\
	Sbucciare le patate (meglio farlo prima, se sono sbucciate si
	possono metter in acqua fredda per evitare che si anneriscono).
	Dopo 30 minuti che la teglia è in forno aggiungere le patate
	tagliate a pezzetti.\\
	Cuocere per 90 minuti evitando che si secchi troppo, nel caso
	aggiungere acqua.\\\\
	Dopo aver messo le patate, mettere una pentola grande riempita
	a metà di acqua salata (stesso sale che per la pasta) sul fuoco.\\\\
	Quando bolle aggiungere pian piano la farina gialla mescolando
	con una frusta, aggiungere un filo d'olio e mescolare a fuoco
	basso per almeno un ora.
\clearpage

\section{Snack salati}

\subsection{Crackers}

\subsubsection{Ingredienti}
\begin{itemize}
\item	500 gr di farina 00
\item	120 gr di olio d'oliva
\item	4-10 gr di lievito di birra
\item	10 gr di sale fino
\item	180 gr d'acqua
\end{itemize}

\subsubsection{Preparazione}
	Sciogliere il lievito di birra in un po' d'acqua, poi unire tutti
	gli ingredienti in una terrina e impastare fino ad ottenere 
	un impasto omogeneo.\\
	Coprire l'impasto e lasciare lievitare 2 ore.\\\\
	Stendere l'impasto con un mattarello più sottile possibile,
	cospargere il piano di lavoro con un po' di farina aiuta a evitare
	che l'impasto si attacchi al tavolo.\\\\
	Posizionare l'impasto su una teglia coperta da carta forno e con
	una rotella da pizza tagliare a strische (lasciandole nella stessa
	posizione, una volta raffreddati si separano lungo il taglio).
	Infornare a forno statico a 200 gradi finchè non sono ben dorati.
\clearpage

\section{Dolci}

\subsection{Biscotti}

\subsubsection{Ingredienti}
\begin{itemize}
\item 300 gr di farina 00
\item	150 gr di burro
\item	200 gr di zucchero
\item	2 uova
\item	5 gr di bicarbonato di sodio
\item	200 gr di goccie di cioccolato
\end{itemize}

\subsubsection{Preparazione}
	Rompere le uova e mescolare tuorli e albumi con una forchetta.\\
	Mischiare tutto in una terrina fino ad ottenere un impasto omogeneo
	(se si tira fuori il burro dal frigo una mezz'ora prima si scalda e 
	si va meglio).\\
	Lasciare riposare l'impasto almeno 30 minuti in frigo (da freddo è
	meno colloso).\\\\
	Disporre l'impasto a palline su una teglia ricoperta di carta forno,
	è comodo usare un cucchiaio per prendere porzioni di impasto e un
	cucchiaino per farlo cadere dal cucchiaio, in forno le palline si
	scioglieranno assumendo la tipica forma schiacciata, anche nel caso
	si unissero si separeranno facilmente una volta raffreddati.\\
	Cuocere in forno statico (senza ventole, simbolo delle 2 resistenze
	sopra e sotto) a 160 gradi finchè non risultano dorati.\\
	Sfornare e lasciar raffreddare, caldi sono ancora umidi e molli.
\clearpage

\subsection{Buche de Noel}

\subsubsection{Ingredienti}
\begin{itemize}
\item   100 gr di farina 00
\item	140 gr di zucchero
\item	5 uova
\item	150 gr di marmellata di fragole
\item	250 gr di panna fresca liquida
\item	200 gr di cioccolato fondente
\item	200 gr di goccie di gioccolato
\end{itemize}

\clearpage
\subsubsection{Preparazione}
	Rompere le uova separando tuorli e albumi, montare a neve gli albumi
	con 50 gr di zucchero in una terrina usando delle fruste elettriche.\\
	In un altra terrina sbattare sempre con le fruste i tuorli con i restanti
	90 gr di zucchero.\\
	Unire i 2 composti e la farina setacciata e mescolare fino a ottenere un impasto
	omogeneo con un movimento dal basso verso l'alto non troppo veloce per evitare di
	smontare gli albumi\\\\
	Stendere l'impasco in una teglia ricoperta di carta forno e infornare a forno statico
	preriscaldato a 220 gradi per 6-7 minuti.\\
	Togliere la teglia dal forno e lasciar raffreddare sigillando l'impasto con pellicola
	trasparente per non far fuoriuscire l'umidità finchè si raffredda.\\\\
	Mettere un pentolino sul fuoco con la panna e portare a bollore, scioglierci il cioccolato
	fondente mescolando con un cucchiaio.\\
	Togliere il pentolino dal fuoco, montare con le fruste il composto e lasciar raffreddare.\\\\
	Dividere l'impasto in 3 fette e creare i seguenti strati dal basso:
	\begin{itemize}
	\item	impasto
	\item   composto di gioccolato e panna
	\item   impasto
	\item   marmellata di fragole
	\item   impasto
	\end{itemize}
	Ricoprire la sommità e i lati di abbondante composto di gioccolato e panna, e poi decorare con
	goccie di gioccolato.\\
	Lasciar riposare il frigorifero almeno 3 ore.
\clearpage

\clearpage
\end{document}
