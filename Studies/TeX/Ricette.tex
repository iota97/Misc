\documentclass[12pt, a4paper]{article}
\usepackage[utf8]{inputenc}
\usepackage[italian]{babel}
\usepackage{fancyhdr}
\usepackage{datetime}

\newdateformat{monthyeardate}{\monthname[\THEMONTH] \THEYEAR}

\def\labelitemi{--}
\setcounter{tocdepth}{2}
\pagestyle{fancy}
\fancyhf{}
\renewcommand{\headrulewidth}{1pt}
\renewcommand{\footrulewidth}{1pt}
\lhead{RICETTE}
\rhead{\leftmark}
\rfoot{Pagina \thepage}

\title{Ricette di cucina}
\author{Giovanni Cocco \thanks{ricette da varie fonti}}
\date{\monthyeardate\today}

\begin{document}

\begin{titlepage}
\maketitle
\thispagestyle{empty}
\end{titlepage}

\begin{abstract}
In questo ricettario verranno collezionate ricette di cucina imparate negli anni.
\end{abstract}
\clearpage

\tableofcontents{}
\clearpage

\section{Primi piatti}

\subsection{Risotto}

\subsubsection{Ingredienti}
\begin{itemize}
\item	1 cipolla o scalogno
\item 	80 g di riso vialone nano a testa
\item 	10 g di parmigiano a testa
\item 	5 g di burro a testa
\item 	1 dado o un cucchiaino di brodo granulato
\item 	10 g di olio di oliva
\end{itemize}

\subsubsection{Preparazione}
	Sciogliere il dado o il brodo granulato in un pentolino d'acqua e
	portarlo a bollore.\\\\
	Tritare la cipolla o scalogno e buttarla assieme all'olio in una 
	pentola, soffriggere la cipolla (lasciare cuocere nell'olio finché 
	non cambia colore mescolando di tanto in tanto).\\\\
	Buttare il riso nel soffritto, mescolare e aggiungere il brodo caldo.
	Continuare a mescolare aggiungendo brodo di tanto in tanto per 
	tenere il livello per 10 minuti.\\\\
	Aggiungere il burro e il parmigiano e cuocere per altri 3-7 minuti,
	aggiungendo brodo o aumentando il fuoco per ottenere la giusta 
	consistenza al termine della cottura.

\clearpage
\subsubsection{Varianti}
\begin{itemize}
\item \emph{Piselli} (70 g a testa sgocciolati) buttarli assieme al riso,
	meglio di dimensioni medie o grandi.
\item \emph{Gorgonzola} (30 g a testa) sostituisce il burro, ma è molto
	più liquido e tende ad allungare il risotto.
\item \emph{Zafferano} aggiungere a 2 minuti dalla fine.
\item \emph{Salsa di pomodoro} (100 g a testa) buttare assieme al riso.
\item \emph{Salsiccia} (1/2 a testa) togliere la pelle e tritarla, buttarla 
	assieme	alla cipolla.
\item \emph{Speck} (50 g a testa) scaldare in un pentolino a parte per
	abbrustolirlo un po' e poi aggiungere assieme al burro.
\item \emph{Zucchini} (1 a testa) cuocere in una pentola con un po, d'acqua
	per 30 minuti per sfaldarli per bene, poi aggiungere 
	assieme	al riso.
\item \emph{Fonduta di parmigiano} (100 ml di panna e 50 g di parmigiano a testa)
	mettere in un pentolino la panna e portarla a bollore, spegnere il fuoco e
	scioglierci dentro il parmigiano a pezzetti mescolando. Disporre la fonduta
	sul fondo del piatto in cui viene servito il risotto.
\end{itemize}
\clearpage

\subsection{Gnocchi di fioretta}

\subsubsection{Ingredienti}
\begin{itemize}
\item	1 kg di fioretta (se è densa ne potrebbe servire di più)
\item	375 g di farina 0
\item 	50 g di pane grattugiato
\item 	15 g di parmigiano (per l'impasto)
\item 	5 g di cannella (per l'impasto)
\item	100 g di burro
\item	10 foglie di salvia
\item	50 g di parmigiano (per il condimento)
\item	10 g di cannella (per il condimento)
\end{itemize}

\subsubsection{Preparazione}
	Mescolare energicamente in una terrina la farina, il pane grattugiato,
	il parmigiano, la cannella e la fioretta fino ad ottenere un impasto
	abbastanza omogeneo. Lasciare riposare l'impasto per un'ora.\\\\
	Mettere sul fuoco una pentola grande di acqua salata (salata come per la pasta, 
	è consigliabile cuocere uno gnocco singolo e poi regolare il sale aggiungendo 
	sale o acqua per compensare). Quando bolle prendere dei cucchiaini di impasto e 
	immergerli nell'acqua così si staccano	da soli. Lasciar cuocere per 5 minuti
	dall'ultimo immerso.\\\\
	Se è fosse necessario cuocere gli gnocchi in più riprese si possono
	mettere in forno statico a 150 gradi per tenerli caldi finché non sono tutti 
	pronti.\\\\
	Mettere fondere il burro in un pentolino con la salvia. Unire il parmigiano e
	la cannella in una ciotola e spolverarci gli gnocchi cotti e poi condire col burro
	fuso.
\clearpage

\subsection{Canederli}

\subsubsection{Ingredienti}
\begin{itemize}
\item	250 g di pane bianco raffermo
\item	150 g di speck
\item 	200 g di latte
\item	10 g di burro
\item 	1/2 cipolla bianca
\item 	2 uova
\item	pan grattato
\item	pepe nero
\item	prezzemolo
\item	erba cipollina
\item	brodo di carne
\item	parmigiano
\end{itemize}

\subsubsection{Preparazione}
	Tritare cipolla e speck, quindi soffriggerli assieme al burro in una padella, poi lasciar raffreddare.\\\\
	Tagliare a dadini il pane, quindi unire il latte, le uova, il prezzemolo, l'erba cipollina, il pepe
	e il soffritto di speck e cipolla.\\\\
	Mescolare l'impasto e aggiungere altro latte o pan grattato per raggiungere la consistenza opportuna,
	quindi formare delle palline del diametro di circa 5 cm.\\\\
	Cuocere i canederli per 15 minuti nel brodo bollente e servire caldi conditi col parmigiano.
\clearpage

\section{Secondi piatti}

\subsection{Spezzatino e polenta}

\subsubsection{Ingredienti}
\begin{itemize}
\item   1.2 kg di muscolo di vitello (polpa di spalla)
\item 	1 kg di patate
\item 	4 cipolle
\item 	2 carote
\item 	1 gambo di sedano
\item 	salvia e rosmarino
\item 	500 g di salsa di pomodoro
\item 	sale e pepe
\item 	1 dado
\item 	100 g di farina 00
\item 	15 g di olio d'oliva
\item 	500 g di farina gialla
\end{itemize}

\clearpage
\subsubsection{Preparazione}
	\paragraph{Spezzatino}\mbox{}\\\\ % New line paragraph hack
	Tritare cipolle, carote e sedano, metterli in una teglia grande
	e mettere a cuocere sul fornello.\\\\
	Dopo 30 minuti aggiungere l'olio, il dado, la salvia e rosmarino
	tritati e mescolare fino a scioglimento del dado.\\\\
	Infarinare la carne con la farina 00 e metterla nella teglia.
	Aggiungere il pomodoro, il sale e il pepe.
	Infornare a forno statico 180 gradi.\\\\
	Sbucciare le patate (meglio farlo prima, se sono sbucciate si
	possono metter in acqua fredda per evitare che si anneriscono).
	Dopo 30 minuti che la teglia è in forno aggiungere le patate
	tagliate a pezzetti.\\\\
	Cuocere per 90 minuti evitando che si secchi troppo, nel caso
	aggiungere acqua.
	\paragraph{Polenta}\mbox{}\\\\
	Mettere una pentola grande riempita
	a metà di acqua salata (stesso sale che per la pasta) sul fuoco.\\\\
	Quando bolle aggiungere pian piano la farina gialla mescolando
	con una frusta, aggiungere un filo d'olio e mescolare a fuoco
	basso per almeno un ora.
\clearpage

\section{Snack salati}

\subsection{Cracker}

\subsubsection{Ingredienti}
\begin{itemize}
\item	500 g di farina 00
\item	120 g di olio d'oliva
\item	4-10 g di lievito di birra
\item	10 g di sale fino
\item	180 g d'acqua
\end{itemize}

\subsubsection{Preparazione}
	Sciogliere il lievito di birra in un po' d'acqua, poi unire tutti
	gli ingredienti in una terrina e impastare fino ad ottenere 
	un impasto omogeneo.\\\\
	Coprire l'impasto e lasciare lievitare 2 ore.\\\\
	Stendere l'impasto con un mattarello più sottile possibile,
	cospargere il piano di lavoro con un po' di farina aiuta a evitare
	che l'impasto si attacchi al tavolo.\\\\
	Posizionare l'impasto su una teglia coperta da carta forno e con
	una rotella da pizza tagliare a strisce (lasciandole nella stessa
	posizione, una volta raffreddati si separano lungo il taglio).\\\\
	Infornare a forno statico a 200 gradi finché non sono ben dorati.
\clearpage

\subsection{Focaccia}

\subsubsection{Ingredienti}
\begin{itemize}
\item	550 g di farina 0
\item	170 g di semola rimacinata di grano duro
\item	65 g di olio d'oliva (per l'impasto)
\item	180 g di latte
\item	1 bustina di lievito di birra
\item 	180 g d'acqua
\item	2 cucchiaini di sale fino
\item	2 cucchiaini di zucchero
\item	olio d'oliva (per l'emulsione)
\item	sale grosso (per decorare)
\end{itemize}

\subsubsection{Preparazione}
	Sciogliere il lievito di birra nell'acqua, poi unire la farina, la semola,
	lo zucchero, il sale fino, i 65 g d'olio e il latte.\\\\
	Mescolare fino ad ottenere una palla e lasciare lievitare 1 ora in una
	terrina coperta con uno strofinaccio umido.\\\\ 
	Stendere l'impasto con le mani in una teglia da forno, premere a fondo con
	le dita per fare dei buchi "decorativi" e cospargere con un emulsione di olio
	e acqua (per emulsionare facilmente si può usare un dispositivo per schiumare 
	il latte). Ricoprire con qualche pizzico di sale grosso e lasciare lievitare
	30 minuti nella teglia coperta con uno strofinaccio umido.\\\\
	Infornare per 20-25 minuti a 190 gradi in forno statico. Tagliare a fette e servire.
\clearpage

\section{Dolci}

\subsection{Biscotti}

\subsubsection{Ingredienti}
\begin{itemize}
\item 	300 g di farina 00
\item	150 g di burro
\item	200 g di zucchero
\item	2 uova
\item	5 g di bicarbonato di sodio
\item	200 g di gocce di cioccolato
\end{itemize}

\subsubsection{Preparazione}
	Rompere le uova e mescolare tuorli e albumi con una forchetta e
	mischiare tutti gli ingredienti in una terrina fino ad ottenere un 
	impasto omogeneo (se si tira fuori il burro dal frigo una mezz'ora prima 
	si scalda e si va meglio).\\\\
	Lasciare riposare l'impasto almeno 30 minuti in frigo (da freddo è
	meno colloso).\\\\
	Disporre l'impasto a palline su una teglia ricoperta di carta forno,
	è comodo usare un cucchiaio per prendere porzioni di impasto e un
	cucchiaino per farlo cadere dal cucchiaio, in forno le palline si
	scioglieranno assumendo la tipica forma schiacciata, anche nel caso
	si unissero si separeranno facilmente una volta raffreddati.\\\\
	Cuocere in forno statico (senza ventole, simbolo delle 2 resistenze
	sopra e sotto) a 160 gradi finché non risultano dorati.\\\\
	Sfornare e lasciar raffreddare, caldi sono ancora umidi e molli.
\clearpage

\subsection{Buche de Noel}

\subsubsection{Ingredienti}
\begin{itemize}
\item   100 g di farina 00
\item	140 g di zucchero
\item	5 uova
\item	150 g di marmellata di fragole
\item	250 g di panna fresca liquida
\item	200 g di cioccolato fondente
\item	200 g di gocce di cioccolato
\end{itemize}

\clearpage
\subsubsection{Preparazione}
	Rompere le uova separando tuorli e albumi, montare a neve gli albumi
	con 50 g di zucchero in una terrina usando delle fruste elettriche.\\\\
	In un altra terrina sbattere sempre con le fruste i tuorli con i restanti
	90 g di zucchero.\\\\
	Unire i 2 composti e la farina setacciata e mescolare fino a ottenere un impasto
	omogeneo con un movimento dal basso verso l'alto non troppo veloce per evitare di
	smontare gli albumi\\\\
	Stendere l'impasto in una teglia ricoperta di carta forno e infornare a forno statico
	preriscaldato a 220 gradi per 6-7 minuti.\\\\
	Togliere la teglia dal forno e lasciar raffreddare sigillando l'impasto con pellicola
	trasparente per non far fuoriuscire l'umidità finché si raffredda.\\\\
	Mettere un pentolino sul fuoco con la panna e portare a bollore, scioglierci il cioccolato
	fondente mescolando con un cucchiaio.\\\\
	Togliere il pentolino dal fuoco, montare con le fruste il composto e lasciar raffreddare.\\\\
	Dividere l'impasto in 3 fette e creare i seguenti strati dal basso:
	\begin{itemize}
	\item	impasto
	\item   composto di cioccolato e panna
	\item   impasto
	\item   marmellata di fragole
	\item   impasto
	\end{itemize}
	Ricoprire la sommità e i lati di abbondante composto di cioccolato e panna, e poi decorare con
	gocce di cioccolato e lasciare riposare in frigorifero per almeno 3 ore.
\clearpage

\subsection{Tiramisù}

\subsubsection{Ingredienti}
\begin{itemize}
\item   300 g panna fresca liquida
\item	300 g di mascarpone
\item	125 g di zucchero a velo
\item	300 g di savoiardi
\item	cacao amaro in polvere
\item	caffè in polvere
\end{itemize}

\subsubsection{Preparazione}
	Fare il caffè con una moka da 4-6 persone e lasciar raffreddare.\\\\
	Montare la panna fresca con metà dello zucchero in una terrina poi
	unire il mascarpone e il restante zucchero a velo e mescolare per ottenere
	una crema uniforme.\\\\
	Versare il caffè in un piatto fondo e disporre uno strato di savoiardi
	inzuppati (1 secondo al massimo) in una teglia, coprire con la crema
	al mascarpone, poi un altro strato di savoiardi inzuppati e coprire
	anche quest'ultimo con la restante crema.\\\\
	Spolverare con il cacao in polvere e riporre in frigorifero per almeno 3 ore.	
\clearpage

\subsection{Pancake}

\subsubsection{Ingredienti}
\begin{itemize}
\item   200 g di farina 00 
\item	250 g di latte
\item	50 g di zucchero
\item	2 uova
\item	5 g di lievito per dolci
\item	burro per ungere la padella
\end{itemize}

\subsubsection{Preparazione}
	Rompere le uova, mescolare tuorli e albumi con una forchetta e unire
	tutti gli ingredienti nel ordine: zucchero, latte, farina setacciata 
	e lievito.\\\\
	Mescolare l'impasto con una frusta fino ad ottenere una pastella
	liscia e senza grumi.\\\\
	Mettere una padella su fuoco medio con una noce di burro una volta
	calda con l'aiuto di un mestolo versare la pastella.\\\\
	Lasciare cuocere qualche secondo, quando si formano parecchie
	bollicine sulla superficie girare il pancake con una spatola
	e far cuore qualche altro secondo.\\\\
	Impilare i pancake cotti in un piatto man mano che vengono cotti e poi
	servire con marmellata, miele o nutella a piacere.
\clearpage

\subsection{Creme caramel}

\subsubsection{Ingredienti}
\begin{itemize}
\item   4 uova
\item	500 g di latte
\item	175 g di zucchero per il creme caramel
\item	2 cucchiaini di rum
\item	75 g di zucchero per il caramello
\item	qualche goccia di succo di limone
\end{itemize}

\subsubsection{Preparazione}
	\paragraph{Caramello}\mbox{}\\\\
	Sciogliere lo zucchero in 25 ml di acqua con qualche goccia di succo di limone
	e mettere sul fuoco, senza mescolare attendere che il caramello diventi di colore
	ambrato non troppo scuro.\\\\
	Togliere dal fuoco e lasciare che il calore rimasto 
	completi la cottura del caramello.
	\paragraph{Creme caramel}\mbox{}\\\\
	Mettere a bollire il latte in un pentolino.\\\\
	Sbattere le uova e lo zucchero in una terrina e aggiungere pian piano il
	latte caldo mescolando e infine il rum.\\\\
	Versare in una pirofila il caramello e poi sopra il composto ancora caldo e
	mettere tutto in forno statico a bagno maria a 150 gradi per 1 ora circa, finché
	il composto non assumerà la consistenza di un budino.\\\\
	Servire rovesciando la pirofila su un piatto in modo a avere il caramello sopra.
\clearpage

\subsection{Muffin}

\subsubsection{Ingredienti}
\begin{itemize}
\item   300 g di farina 00
\item	100 g di zucchero
\item	1 bustina di lievito
\item	1 cucchiaino di bicarbonato
\item	200 g di latte
\item	2 uova
\item	125 g di burro
\end{itemize}

\subsubsection{Preparazione}
	Rompere le uova in una terrina e sbattere con una forchetta per unire tuorlo e 
	albume, aggiungere nell'ordine: la farina, lo zucchero, il lievito, il bicarbonato,
	il latte e il burro.\\\\
	Mescolare l'impasto con una frusta fino a ottenere un impasto omogeneo, ma che
	presenti ancora qualche piccolissimo grumo.\\\\
	Mescolato l'impasto è possibile creare varianti aggiungendo ad esempio gocce di
	cioccolato, mirtilli, mele o pere a cubetti, etc...\\\\
	Disporre l'impasto in degli stampini imburrati e infornare a forno statico 180
	gradi per 15-20 minuti.\\\\
	Inserendo a metà stampino quando viene riempito una pallina di nutella congelata,
	mettendola in congelatore preventivamente è possibile ottenere un cuore di nutella.
	
\clearpage

\subsection{Salame di cioccolato}

\subsubsection{Ingredienti}
\begin{itemize}
\item	200 g di cioccolato fondente
\item	200 g di biscotti novellino
\item	100 g di zucchero
\item	50 g di burro
\item   2 uova
\end{itemize}

\subsubsection{Preparazione}
	Sciogliere il cioccolato in un pentolino a bagno maria.\\\\
	In una terrina unire le uova, il burro e lo zucchero e montare con le
	fruste elettriche.\\\\
	Unire il cioccolato fuso e montare ancora un po' con le fruste fino
	a ottenere un impasto omogeneo.\\\\
	Sbriciolare in pezzetti i biscotti e unirli all'impasto mescolando
	il tutto con un cucchiaio.\\\\
	Stendere l'impasto nella carta forno e dargli una forma a salame arrotolando
	la carta forno attorno all'impasto.\\\\
	Lasciar riposare in frigo 4-6 ore.
\clearpage

\subsection{Mousse di cioccolato}

\subsubsection{Ingredienti}
\begin{itemize}
\item   4 uova
\item	100 g di cioccolato fondente
\item	50 g di zucchero
\item	un pizzico di sale
\end{itemize}

\subsubsection{Preparazione}
	Sciogliere il cioccolato in un pentolino a bagno maria.\\\\
	Separare i tuorli dagli albumi e montare a neve ferma gli albumi con 25 g di
	zucchero e un pizzico di sale.\\\\
	Montare i tuorli con i restanti 25 g di zucchero.\\\\
	Unire tuorli, albumi e il cioccolato fuso, mescolare con un movimento dal basso verso l'alto.\\\\
	Lasciar riposare in frigo almeno 2 ore.
\clearpage

\subsection{Tarte au citron}

\subsubsection{Ingredienti}
\begin{itemize}
\item   250 g di farina 00
\item	200 g di burro
\item	530 g di zucchero
\item	7 uova
\item	3 limoni
\item	5 g di amido di mais
\item	un pizzico di sale
\end{itemize}

\subsubsection{Preparazione}
Unire 100 g di burro, 250 g di farina, 130 g di zucchero, 2 uova sbattute e 5 g di scorza di limone in
una terrina, lavorare fino ad ottenere un impasto liscio. Avvolgere nella pellicola e lasciar riposare in frigo
1 ora.\\\\
Spremere i 3 limoni ed estrarre circa 100 g di succo a cui unire i 5 g di amido di mais.\\\\
In un pentolino a bagno maria scogliere 100 g di burro e 200 g di zucchero. Poi aggiungere il succo di limone con
l'amido e 2 uova sbattute.\\\\
Continuare a mescolare senza far bollire il composto. Una volta addensato filtrarlo con un colino e 
lasciar raffreddare.\\\\
Stendere la frolla in una tortiera, bucare il fondo con una forchetta e infornare a forno statico 180 gradi per circa 30 minuti.\\\\
Montare 3 albumi con 20 g di zucchero e un pizzico di sale. Mettere sul fuoco un pentolino con 50 g di acqua e 180 g di zucchero e
portare a bollore per un minuto senza che si caramelli. Unire lo sciroppo di zucchero ancora caldo agli albumi montati e montare con le
fruste a velocità massima.\\\\
Mettere la lemon curd nella pasta frolla, poi con l'aiuto di una sac a poche decorare la sommità con la meringa. Porre in forno modalità
grill per un paio di minuti.
\clearpage
\end{document}
